\documentclass[11pt]{article}
\usepackage{amsmath}
\usepackage{amssymb}
                             % The preamble begins here.
\title{School of Mathematics and Statistics
\\ MAST90026
Computational Differential Equations \\ 2024
} 
% Declares the document's title.
%\author{Leslie Lamport}      % Declares the author's name.
\date{}      % Deleting this command produces today's date.

\textwidth=17.02cm
\textheight=25cm
\voffset=-3cm
\hoffset=-0.75in
\oddsidemargin 0.75in
\evensidemargin 0.75in
% Remove paragraph indent
\setlength{\parindent}{0pt}

%\renewcommand{\theenumi}{\alph{enumi}}
\renewcommand{\theenumii}{\roman{enumii}}

\newenvironment{pseudo}{\ttfamily \begin{tabbing} }{\end{tabbing}}

%\setcounter{section}{0}


\begin{document}             % End of preamble and beginning of text.

\maketitle                   % Produces the title.

\part*{ Homework 2 \\ Due: 11:00AM Wednesday, 13th March.}   

This homework is worth 5\% of the total assessment in this subject.  
You should submit copies of {\sc Matlab}  programs (include all files necessary for the programs to run) 
and sufficient relevant output online through LMS (You may find the Matlab command publish useful!). Any hand written working should be scanned and converted to a PDF. 
\medskip

All files should be compressed into a single zip file \emph{with your student ID number in the file name}.



\begin{enumerate}
\item 
Write a code  to solve the mixed BC problem
\[ u'' + p u' + q u = r;\ u(a) - u'(a)= \alpha,\ u(b) = \beta,    \] 
 where $p, q, r, a, b, \alpha, \beta$ are constants.
 
 Use
  \begin{itemize}
\item [(i)] a first order FD formula
\item [(ii)] a 2nd order method
\end{itemize}
to handle the Robin BC at $x=a$.

Test your code on problem 
\[ u'' - u = 0;\ \ u(0)-u'(0) = 0 , u(1)  = \exp(1),  \] 
 whose exact solution is $u(x) = \exp(x) $ and plot the maximum grid error $\max |e_j|$ versus $N$ 
as a log-log plot, for each method.

%\item  Solve problem A from Homework 1 by hand using collocation with 1 collocation point at $x=1/2$ and the 2 BCs 
%(3 conditions) to determine the polynomial of order 3 (quadratic)
% \[ u(x) \approx c_0+c_1 x+c_2 x^2. \]
% Plot your approximate answer versus the exact answer.
 
 
\item  Write code to use collocation at Legendre Gaussian Lobatto points to solve the constant coefficient Dirichlet BVP:
\[ u'' + p u' + q u = r;\ \ u(-1) = \alpha, u(1) = \beta,   \] 
 where $p, q, r, \alpha, \beta$ are constants. You may use the file `\verb+cheb.m+'. 

\begin{itemize}
	\item [(a)]
 Test your code on   problem A:
\[ u'' - u = 0;\ \ u(-1) = 1, u(1) = 3,   \] 
  whose exact solution is $\hat{u}(x) = 2\cosh (x) /\cosh (1)+\sinh (x) /\sinh (1)$. Inspect the solution visually by plotting both the numerical and exact solution on the same axes.
  
  \item[(b)] Approximate the error using $\lVert E\rVert_2 = \sqrt{\int_{-1}^{1} (u - \hat{u})^2dx }$ and an appropriate quadrature rule. Plot the error vs $N$. 
 

 \item [(b)]  What change would you have to make to handle the problem with  
 $u(a)=\alpha, u(b)=\beta$.
 \end{itemize} 
\end{enumerate}


 
\end{document}