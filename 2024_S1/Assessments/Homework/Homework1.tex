\documentclass[11pt]{article}
\usepackage{amsmath}
\usepackage{amssymb}
                             % The preamble begins here.
\title{School of Mathematics and Statistics
\\ MAST90026
Computational Differential Equations \\ 2024
} 
% Declares the document's title.
%\author{Leslie Lamport}      % Declares the author's name.
\date{}      % Deleting this command produces today's date.

\textwidth=17.02cm
\textheight=25cm
\voffset=-3cm
\hoffset=-0.75in
\oddsidemargin 0.75in
\evensidemargin 0.75in
% Remove paragraph indent
\setlength{\parindent}{0pt}

%\renewcommand{\theenumi}{\alph{enumi}}
\renewcommand{\theenumii}{\roman{enumii}}

\newenvironment{pseudo}{\ttfamily \begin{tabbing} }{\end{tabbing}}

%\setcounter{section}{0}


\begin{document}             % End of preamble and beginning of text.

\maketitle                   % Produces the title.

\part*{ Homework 1 \\ Due: 11:00AM Wednesday, 6th March.}   

This homework is worth 5\% of the total assessment in this subject.  
You should submit copies of {\sc Matlab}  programs (include all files necessary for the programs to run) 
and sufficient relevant output online through LMS (You may find the Matlab command publish useful!). Any hand written working should be scanned and converted to a PDF. 
\medskip

All files should be compressed into a single zip file \emph{with your student ID number in the file name}.



\begin{enumerate}
\item 
Consider the matrix $A\in \mathbb{R}^{N\times N}$ which is  obtained by using central finite difference method to discretize $u''=f(x), u(0) = u(1) = 0$, i.e.
\begin{equation}
	A= \frac{1}{h^2}\begin{bmatrix}
		-2 & 1 & & & &\\
		1& -2 & 1 & & &\\
		& & \ddots & \ddots & \ddots & &\\
		& & & 1 & -2 &1\\
		& & & & 1 & -2 &\\
	\end{bmatrix}
\end{equation}
\begin{itemize}
	\item [(a)] Denote the $k$th eigenvalue of A by $\lambda^k$ and the corresponding eigenvalue vector is denoted by $\mathbf{u}^k$. Find $\lambda^k$ by assuming  $\mathbf{u}^k_j = \sin(\frac{k\pi j}{N+1})$, $j=1, \ldots, N;$  
	\item [(b)] Use the result in (a) to prove that $\|A^{-1}\|_2 = \frac{1}{\pi^2} + O(h^2)$.
\end{itemize}

\item  Consider the finite difference scheme for the 1D steady state convection-diffusion equation
$$
\begin{aligned}
&\epsilon u^{\prime \prime}-u^{\prime}=-1, \quad 0<x<1 \\
&u(0)=1, \quad u(1)=3
\end{aligned}
$$
(a) Verify the exact solution is
$$
u(x)=1+x+\left(\frac{e^{x / \epsilon}-1}{e^{1 / \epsilon}-1}\right)
$$
(b) Compare the following two finite difference methods for $\epsilon=0.3,0.1,0.05$, and $0.0005$.
\begin{itemize}
	\item [(i)] Central finite difference scheme:
$$
\epsilon \frac{U_{i-1}-2 U_{i}+U_{i+1}}{h^{2}}-\frac{U_{i+1}-U_{i-1}}{2 h}=-1
$$
\item [(ii)] Central-upwind finite difference scheme:
$$
\epsilon \frac{U_{i-1}-2 U_{i}+U_{i+1}}{h^{2}}-\frac{U_{i}-U_{i-1}}{h}=-1
$$
\end{itemize}
Do a grid refinement analysis for each case to determine the order of accuracy. Include plots of $\|\boldsymbol{E}\|_2$ vs $h$.

(c) From your observations, in your opinion which method is better? 

\end{enumerate}


 
\end{document}