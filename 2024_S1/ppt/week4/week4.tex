\documentclass[10pt]{beamer}

\usetheme{metropolis}
\usepackage{appendixnumberbeamer}
\usepackage{booktabs}
\usepackage[scale=2]{ccicons}
\usepackage{pgfplots}
\usepackage{color}
\usepgfplotslibrary{dateplot}
\usepackage{xspace}
\newcommand{\themename}{\textbf{\textsc{metropolis}}\xspace}

\usepackage{stmaryrd}
\usepackage{amsmath}
\usepackage{verbatim}
\usepackage{tikz}
\usepackage{siunitx}
\usepackage{color}
\usepackage[normalem]{ulem}
\usetikzlibrary{calc,decorations.pathmorphing,patterns}
\usepackage{amssymb}
\usepackage{amsfonts}
\usepackage{amscd}
\usepackage{amsthm}
\usepackage{mathrsfs}
\usepackage{enumerate}
\usepackage{mathtools}
\usepackage{booktabs}
\usepackage{array}
\usepackage{nth}
\usepackage{lipsum}


%%%%%%%%%%%%%%%%%%%%%%%%%%%%%list %%%%%%%%%%%%%%%%%
\usepackage{textcomp}           % To use with matlab-prettifier
\usepackage{listings}
\usepackage[framed , numbered]{matlab-prettifier}
\usepackage{listings}
%%%%%%%%%%%%%%%%%end of list%%%%%%%%%%%%%%%%%%%%%

%%%%%%%%%%%%%%%%%%%%%%%%%%%%%%%%%%%%%%%%%%%%%%%%%%%%%%%%%%%%%%%%%%%%%%%%%
%%%% The following is for fancy box %%%%%%%%%%%%%%
\usepackage[framemethod=TikZ]{mdframed}
%%%%%%%%%%%%% Theorem %%%%%%%%%%%%
\newenvironment{thm}[2][]{%
\ifstrempty{#1}%
{\mdfsetup{%
frametitle={%
\tikz[baseline=(current bounding box.east),outer sep=0pt]
\node[anchor=east,rectangle,fill=red!20]
{\strut Theorem};}}
}%
{\mdfsetup{%
frametitle={%
\tikz[baseline=(current bounding box.east),outer sep=0pt]
\node[anchor=east,rectangle,fill=red!20]
{\strut Theorem #1};}}%
}%
\mdfsetup{innertopmargin=1pt,linecolor=red!20,%
linewidth=2pt,topline=true,%
frametitleaboveskip=\dimexpr-\ht\strutbox\relax
}
\begin{mdframed}[]\relax%
\label{#2}}{\end{mdframed}}
%%%%%%%%%%%%%%Lemma%%%%%%%%%%%%%%%%%%%%%%%%%%%%%%%%%%%%%%%%%%%%%%%%%%%%%%
\newcounter{lem}[section] \setcounter{lem}{0}
\renewcommand{\thelem}{}
\newenvironment{lem}[2][]{%
\refstepcounter{lem}%
\ifstrempty{#1}%
{\mdfsetup{%
frametitle={%
\tikz[baseline=(current bounding box.east),outer sep=0pt]
\node[anchor=east,rectangle,fill=green!20]
{\strut Lemma~\thelem};}}
}%
{\mdfsetup{%
frametitle={%
\tikz[baseline=(current bounding box.east),outer sep=0pt]
\node[anchor=east,rectangle,fill=green!20]
{\strut Lemma~\thelem:~#1};}}%
}%
\mdfsetup{innertopmargin=1pt,linecolor=green!20,%
linewidth=2pt,topline=true,%
frametitleaboveskip=\dimexpr-\ht\strutbox\relax
}
\begin{mdframed}[]\relax%
\label{#2}}{\end{mdframed}}
%%%%%%%%%%%%%%%%%%%%%%%%%%%%%%%%%%%%%%%%%%%%%%%%%%%%%%%%%%%%%%%%%%%%%%%
%Corollary
\newenvironment{cor}[2][]{%
\ifstrempty{#1}%
{\mdfsetup{%
frametitle={%
\tikz[baseline=(current bounding box.east),outer sep=0pt]
\node[anchor=east,rectangle,fill=blue!30]
{\strut Corollary};}}
}%
{\mdfsetup{%
frametitle={%
\tikz[baseline=(current bounding box.east),outer sep=0pt]
\node[anchor=east,rectangle,fill=blue!30]
{\strut Corollary:~#1};}}%
}%
\mdfsetup{innertopmargin=1pt,linecolor=blue!30,%
linewidth=2pt,topline=true,%
frametitleaboveskip=\dimexpr-\ht\strutbox\relax
}
\begin{mdframed}[]\relax%
\label{#2}}{\end{mdframed}}
%%%%%%%%%%%%%%%%%%%%%%%%%%%%%%%%%%%%%%%%%%%%%%%%%%%%%%%%%%%%%%%%%%%%%%%
%Proof
\newenvironment{prove}[2][]{%
\ifstrempty{#1}%
{\mdfsetup{%
frametitle={%
\tikz[baseline=(current bounding box.east),outer sep=0pt]
\node[anchor=east,rectangle,fill=red!20]
{\strut Proof};}}
}%
{\mdfsetup{%
frametitle={%
\tikz[baseline=(current bounding box.east),outer sep=0pt]
\node[anchor=east,rectangle,fill=red!20]
{\strut Proof:~#1};}}%
}%
\mdfsetup{innertopmargin=1pt,linecolor=red!20,%
linewidth=2pt,topline=true,%
frametitleaboveskip=\dimexpr-\ht\strutbox\relax
}
\begin{mdframed}[]\relax%
\label{#2}}{\end{mdframed}}
%%%%%%%%%%%%%%%%%%%%%%%%%%%%%%%%%%%%%%%%%%%%%%%%%%%%%%%%%%%%%%%%%%%%%%%
%Defintion
\newcounter{drf}[section]\setcounter{drf}{0}
\renewcommand{\thedrf}{}
\newenvironment{drf}[2][]{%
\refstepcounter{drf}%
\ifstrempty{#1}%
{\mdfsetup{%
frametitle={%
\tikz[baseline=(current bounding box.east),outer sep=0pt]
\node[anchor=east,rectangle,fill=orange!20]
{\strut Definition~\thedrf};}}
}%
{\mdfsetup{%
frametitle={%
\tikz[baseline=(current bounding box.east),outer sep=0pt]
\node[anchor=east,rectangle,fill=orange!20]
{\strut Definition~\thedrf:~#1};}}%
}%
\mdfsetup{innertopmargin=1pt,linecolor=orange!20,%
linewidth=2pt,topline=true,%
frametitleaboveskip=\dimexpr-\ht\strutbox\relax
}
\begin{mdframed}[]\relax%
\label{#2}}{\end{mdframed}}
%%%%%%%%%%%%%%%%%%%%%%%%%%%%%%%%%%%%%%%%%%%%%%%%%%%%%%%%%%%%%%%%%%%%%%%%%%%
% end of self defined label
%%%%%%%%%%%%%%%%%%%%%%%%%%%%%%%%%%%%%%%%%%%%%%%%%%%%%%%%%%%%%%%%%%%%%%%%%%%


%%%%%%%%%%%%%% For double curly bracket%%%%%%%%%%%%%%%%
\usepackage{xparse}

\NewDocumentCommand{\dgal}{sO{}m}{%
  \IfBooleanTF{#1}
    {\dgalext{#3}}
    {\dgalx[#2]{#3}}%
}

\NewDocumentCommand{\dgalext}{m}{%
  \sbox0{%
    \mathsurround=0pt % just for safety
    $\left\{\vphantom{#1}\right.\kern-\nulldelimiterspace$%
  }%
  \sbox2{\{}%
  \ifdim\ht0=\ht2
    \{\kern-.45\wd2 \{#1\}\kern-.45\wd2 \}%
  \else
    \left\{\kern-.5\wd0\left\{#1\right\}\kern-.5\wd0\right\}%
  \fi
}

\NewDocumentCommand{\dgalx}{om}{%
  \sbox0{\mathsurround=0pt$#1\{$}%
  \sbox2{\{}%
  \ifdim\ht0=\ht2
    \{\kern-.45\wd2 \{#2\}\kern-.45\wd2 \}%
  \else
    \mathopen{#1\{\kern-.5\wd0 #1\{}
    #2
    \mathclose{#1\}\kern-.5\wd0 #1\}}
  \fi
}
%%%%%%%%%%%%%%%%



%%%%%%%%%%%%%%%%%%%%%%%%%%%%%%%%%%%%%

\usepackage{animate}

\graphicspath{{./fig/}}


%%%%%%%%%%%%%% define color %%%%%%%%%%%%%%%%%%
\definecolor{DarkFern}{HTML}{407428}
\definecolor{DarkCharcoal}{HTML}{4D4944}
\colorlet{Fern}{DarkFern!85!white}
\colorlet{Charcoal}{DarkCharcoal!85!white}
\colorlet{LightCharcoal}{Charcoal!50!white}
\colorlet{AlertColor}{orange!80!black}
\colorlet{DarkRed}{red!70!black}
\colorlet{DarkBlue}{blue!70!black}
\colorlet{DarkGreen}{green!70!black}

\newcommand{\I}{\mathrm{i}}




%%%%%%%%%%%%%%%%%%%%%%%%%%%%%%%%%%%%%%
%\newcommand{\G}{\alert{G}}  % change color of main author



%%%%%%%%%%%%% Title Page %%%%%%%%%%%%%%%%%%%

\title{MAST90026 Computational Differential\\Equations: Week 4}
%\subtitle{A modern beamer theme}
\date{Semester 1 2024}
\author{Jesse Collis\\Modified from Hailong Guo (2022)}
\institute{The University of Melbourne}
\titlegraphic{\vspace{6cm}\flushright\includegraphics[height=2cm]{logo.png}}


\begin{document}

\maketitle


%
%
%\begin{frame}{Table of contents}
%  \setbeamertemplate{section in toc}[sections numbered]
%  \tableofcontents[hideallsubsections]
%\end{frame}


%-=-=-=-=-=-=-=-=-=-=-=-=-=-=-=-=-=-=-=-=-=-=-=-=
%
%	SECTION: 
%
%-=-=-=-=-=-=-=-=-=-=-=-=-=-=-=-=-=-=-=-=-=-=-=-=
%\section{Method of Weighted Residuals}





%-=-=-=-=-=-=-=-=-=-=-=-=-=-=-=-=-=-=-=-=-=-=-=-=
%	FRAME:
%-=-=-=-=-=-=-=-=-=-=-=-=-=-=-=-=-=-=-=-=-=-=-=-=
\begin{frame}{Handling nonlinearities}



\resizebox{.85\linewidth}{!}{
  \begin{minipage}{\linewidth}
\begin{equation*}
\begin{CD}
\mbox{Non-linear BVP} @>\mbox{discretize}>\mbox{FDM, collocation, FEM}> \mbox{Non-linear algebraic equations}\\
@V\mbox{quasilinearise}VV @VV\mbox{linearise}V\\
\mbox{Sequence of linear BVPs} @>\mbox{discretize}>> \mbox{Solve a sequence of linear systems}
\end{CD}
\end{equation*}
  \end{minipage}
}


\end{frame}



%-=-=-=-=-=-=-=-=-=-=-=-=-=-=-=-=-=-=-=-=-=-=-=-=
%	FRAME:
%-=-=-=-=-=-=-=-=-=-=-=-=-=-=-=-=-=-=-=-=-=-=-=-=
\begin{frame}{Demonstration }
Consider a simple non-linear BVP:
\begin{equation*}
u''+1-u^2=0, \qquad u(0)=0, u(1)=1,
\end{equation*}
with $N+1$ mesh points or $N-1$ internal points.

\vspace{1em}

FDM:
\begin{equation*}
\frac{u_{j-1}-2u_j+u_{j+1}}{h^2} +1-u^2_j=0, \qquad j=2,\dots N, \qquad u_1=0,\,u_{N+1}=1.
\end{equation*}



\end{frame}
   

%-=-=-=-=-=-=-=-=-=-=-=-=-=-=-=-=-=-=-=-=-=-=-=-=
%	FRAME:
%-=-=-=-=-=-=-=-=-=-=-=-=-=-=-=-=-=-=-=-=-=-=-=-=
\begin{frame}{Matrix form of FDM }

System of nonlinear equations:
\begin{equation*}
\frac{1}{h^2} 
\begin{pmatrix}
-2 & 1 & 0 & \cdots & \cdots \\
1 & -2 & 1 & 0 & \vdots \\
0 & 1& \ddots & \ddots & \vdots\\
\vdots & \vdots & \ddots & \ddots & \vdots\\
\vdots & \cdots &\cdots &\cdots &-2
\end{pmatrix}
\begin{pmatrix}
u_2 \\
\vdots\\
\vdots\\
\vdots\\
u_N
\end{pmatrix}
+
\begin{pmatrix}
1-u_2^2 \\
\vdots\\
\vdots\\
\vdots\\
1-u_N^2
\end{pmatrix}
=
\begin{pmatrix}
0 \\
0\\
\vdots\\
\vdots\\
-\frac{1}{h^2}
\end{pmatrix}
\end{equation*}



\vspace{1em}

This is a nonlinear system of algebraic equations
 \[\vec{F}(\vec{u})=A\vec{u} + 1-\vec{u}\odot\vec{u}-\vec{b}=\vec{0}.\]
\end{frame}
   



%-=-=-=-=-=-=-=-=-=-=-=-=-=-=-=-=-=-=-=-=-=-=-=-=
%	FRAME:
%-=-=-=-=-=-=-=-=-=-=-=-=-=-=-=-=-=-=-=-=-=-=-=-=
\begin{frame}{Method for solving nonlinear equations }

Fixed point or Picard iteration (easy to implement but usually linearly convergent)
 \vspace{1em}
 
 Newton's Method (need $f'$ but quadratically convergent)
  
  \vspace{1em}
  
  Bisection (globally convergent but slow)
  
  \vspace{1em}
  
  Secant method (don't need $f'$ but super linear convergence)

\end{frame}




%-=-=-=-=-=-=-=-=-=-=-=-=-=-=-=-=-=-=-=-=-=-=-=-=
%	FRAME:
%-=-=-=-=-=-=-=-=-=-=-=-=-=-=-=-=-=-=-=-=-=-=-=-=
\begin{frame}{Fixed point (Picard) iteration }

Idea of Picard iteration: rewrite $f(x)=0$ as $x = g(x)$.

  \vspace{1em}

Condition on convergence: $g$ is contractive i.e. $|g(x)-g(y)|< C|x-y|$ for some $C<1$. 


\vspace{1em}



Sufficient condition: $|g'|<1$. 



\vspace{1em}


Higher-dimensional: rewrite $\vec{F}(\vec{u})=\vec{0}$ as $\vec{u}=\vec{G}$

 \vspace{1em}
 
Our example:  \[ \vec{u}_{N+1} = \vec{A}^{-1} [\vec{b} - (1- \vec{u}_N\odot\vec{u}_N)].\]

\end{frame}


   


%-=-=-=-=-=-=-=-=-=-=-=-=-=-=-=-=-=-=-=-=-=-=-=-=
%	FRAME:
%-=-=-=-=-=-=-=-=-=-=-=-=-=-=-=-=-=-=-=-=-=-=-=-=
\begin{frame}{Idea of Newton's method }

    \begin{center}
  \includegraphics[width=0.8\textwidth]{newtongraph}
     \end{center}
     
\end{frame}
   




%-=-=-=-=-=-=-=-=-=-=-=-=-=-=-=-=-=-=-=-=-=-=-=-=
%	FRAME:
%-=-=-=-=-=-=-=-=-=-=-=-=-=-=-=-=-=-=-=-=-=-=-=-=
\begin{frame}{Derivation  of Newton's method using Taylor series}


Using Taylor series:
\[f(x_{n+1}) \approx f(x_n) - f'(x_n)(x_{n+1}-x_n),\]
or written as an update 
\[x_{n+1} = x_n + \Delta x_n, \quad \text{where} \quad  \Delta x_n = \frac{-f(x_n)}{f'(x_n)}.\]
 
 
 
  \vspace{1em}
 Quadratic convergence, i.e. $e_{n+1} \le K e_{n}^2$ but not always convergent.
 
\end{frame}
   




%-=-=-=-=-=-=-=-=-=-=-=-=-=-=-=-=-=-=-=-=-=-=-=-=
%	FRAME:
%-=-=-=-=-=-=-=-=-=-=-=-=-=-=-=-=-=-=-=-=-=-=-=-=
\begin{frame}{2D Newton's method }

Consider  $F(u,v)=0$, and $G(u,v)=0$. 

 \vspace{1em}

Taylor series expansion: 
\begin{align*}
&0=F(u_{n+1},v_{n+1}) \approx F(u_n,v_n) + \frac{\partial F}{\partial u} \bigg|_n(u_{n+1}-u_n) + \frac{\partial F}{\partial v}\bigg|_n (v_{n+1} - v_n),\\
&0=G(u_{n+1},v_{n+1}) \approx G(u_n,v_n) + \frac{\partial G}{\partial u} \bigg|_n\Delta u_n + \frac{\partial G}{\partial v}\bigg|_n \Delta v_n,
\end{align*}


 \vspace{1em}
Matrix form
\begin{equation*}
\begin{pmatrix}
\frac{\partial F}{\partial u}\big|_n & \frac{\partial F}{\partial v}\big|_n\\
\frac{\partial G}{\partial u}\big|_n & \frac{\partial G}{\partial v}\big|_n
\end{pmatrix}
\begin{pmatrix}
\Delta u_n\\
\Delta v_n
\end{pmatrix}
=-
\begin{pmatrix}
F(u_n,v_n)\\
G(u_n,v_n)
\end{pmatrix}
=-
\vec{F}(\vec{u}_n).
\end{equation*}

 
\end{frame}
   
   
   
   

%-=-=-=-=-=-=-=-=-=-=-=-=-=-=-=-=-=-=-=-=-=-=-=-=
%	FRAME:
%-=-=-=-=-=-=-=-=-=-=-=-=-=-=-=-=-=-=-=-=-=-=-=-=
\begin{frame}{General Newton's method }

Derive using Taylor  series expansion:
$$
0=\mathbf{F}\left(\mathbf{u}_{n+1}\right) \approx \mathbf{F}\left(\mathbf{u}_{n}\right)+\mathbf{F}^{\prime}\left(\mathbf{u}_{n}\right)\left(\mathbf{u}_{n+1}-\mathbf{u}_{n}\right) .
$$

 \vspace{1em}
 Newton update
$$
\mathbf{u}_{n+1}=\mathbf{u}_{n}+\delta_{n}
$$
where $\delta_{n}$ solves the linear system
$$
J\left(\mathbf{u}_{n}\right) \delta_{n}=-\mathbf{F}\left(\mathbf{u}_{n}\right)
$$
Here $J(\theta) \equiv F^{\prime}(\theta) \in \mathbb{R}^{m \times m}$ is the Jacobian matrix with elements
$$
J_{i j}(\mathbf{u})=\frac{\partial}{\partial u_{j}} F_{i}(\mathbf{u}),
$$
where $F_{i}(\mathbf{u})$ is the $i$ th component of the vector-valued function $\mathbf{F}$.


 
\end{frame}
      
      
      
      
\begin{frame}{Discussion of Newton's method}

\alert{Pros}: If it convergent, it is quadratic convergent.


 \vspace{1em}
 \alert{Cons}: 
 \begin{itemize}
 	\item  Need  to compute the $N \times N$ matrix $J$ of partial derivative.
 	\item Need to solve a sequence of $N \times N$ linear systems  ($O(N^3)$ work).
 \end{itemize}
\end{frame}




%-=-=-=-=-=-=-=-=-=-=-=-=-=-=-=-=-=-=-=-=-=-=-=-=
%	FRAME:
%-=-=-=-=-=-=-=-=-=-=-=-=-=-=-=-=-=-=-=-=-=-=-=-=
\begin{frame}{Back to our Example }
Considering $u''+1-u^2=0$ again
\begin{align*}
&F_i=\sum_j {A}_{ij}u_j +  (1-u_i^2)-{b}_i=0,\\
&J_{ij} = \frac{\partial F_i}{\partial u_j} = {A}_{ij} - 2u_j\delta_{ij}.
\end{align*}



In this case,  $J$ was sparse (tridiagonal) so it's not too hard to compute or solve with.  It's a good idea to declare it sparse or use \alert{spdiags} to create it.

 
\end{frame}
      

%-=-=-=-=-=-=-=-=-=-=-=-=-=-=-=-=-=-=-=-=-=-=-=-=
%	FRAME:
%-=-=-=-=-=-=-=-=-=-=-=-=-=-=-=-=-=-=-=-=-=-=-=-=
\begin{frame}{Applicability of Newton's method}
Need  $J_{ij}=\frac{\partial F_i}{\partial u_j}$ which is often expensive because it has $N^2$ elements. It is also often hard to evaluate the Jacobian analytically.

\begin{itemize}
	\item  Finite difference approximation for the partial derivative.
\begin{equation*}
\frac{\partial F_i}{\partial u_j} \approx \frac{F_i(u_j + \delta) - F_i(u_j)}{\delta},
\end{equation*}
where $\delta \sim C\sqrt{r}$ where $C$ is a constant and $r$ is the unit roundoff. This retains quadratic convergence down to $e_{n+1} \sim r$.
\item Use \alert{automatic differentiation} (Ref:  Andreas Griewank) which  finds the action of the Jacobian on $F$.
\end{itemize}



 
\end{frame}







%-=-=-=-=-=-=-=-=-=-=-=-=-=-=-=-=-=-=-=-=-=-=-=-=
%	FRAME:
%-=-=-=-=-=-=-=-=-=-=-=-=-=-=-=-=-=-=-=-=-=-=-=-=
\begin{frame}{Efficiency of Newton's method}
Idea: evaluate $J$, which has $N^2$ elements as seldom as possible whilst also solving $J \Delta u = -f$ ($O(N^3)$ operations) as seldom as possible.

\vspace{1em}
Motivation: solve $J \Delta u = -f$ in two steps: 1. factorise $J=LU$ into upper and lower triangular matrices; 2. solve the triangular systems which takes $O(N^2)$ time.
S modify Newton's method to evaluate $J$ only every so often instead of every step.




\begin{itemize}
\item \alert{Chord method}:  start at $\vec{u_0}$ and only evaluate and factorise $J(\vec{u}_0)$ once and then solve 
\[(LU)_0 \Delta u_{n} = -F(\vec{u_n}), \ \ u_{n+1}=u_n+\Delta u_n,\]
 repeatedly. much cheaper but only has linear convergence.


\item Re-evaluate $J$ and refactorise every $m$ step. ($m=1$ is Newton's method and $m=\infty$ is the chord method).
\item \alert{inexact Newton methods} and \alert{truncated Newton methods}: solve the systems approximately using an iterative solver.
\end{itemize}

 
\end{frame}






%-=-=-=-=-=-=-=-=-=-=-=-=-=-=-=-=-=-=-=-=-=-=-=-=
%	FRAME:
%-=-=-=-=-=-=-=-=-=-=-=-=-=-=-=-=-=-=-=-=-=-=-=-=
\begin{frame}{Convergence of Newton's method}
Newton's method only solves equations if $\vec{u}_0$ is close enough to the solution $\vec{u^*}$. \alert{Question}: how do we find $\vec{u}_0$?




\alert{Damped Newton methods} which seek to globalize Newton's method by increasing the basin of attraction, $r$.

To do this, we observe that often the step size is too big causing us to over shoot the root, even though we have the direction right.
To do this we replace the Newton step (or modified Newton step) by:
\begin{equation*}
\vec{u}_{n+1}=\vec{u}_n+\lambda\Delta u_n,
\end{equation*}
where $\lambda \in (0,1)$ is the damping factor.


We can modify this further by varying $\lambda$ from step to step, always choosing $\lambda_n$ such that $\|F\|$ falls by a ``sufficient" amount (Ref: Armijo 1966).


 
\end{frame}









%-=-=-=-=-=-=-=-=-=-=-=-=-=-=-=-=-=-=-=-=-=-=-=-=
%	FRAME:
%-=-=-=-=-=-=-=-=-=-=-=-=-=-=-=-=-=-=-=-=-=-=-=-=
\begin{frame}{Quasi-linearisation}
 
 Consider the boundary value problem:
\begin{align*}
u''+q(u)&=f(x), \quad u(a)=\alpha, u(b)=\beta.
\end{align*}


\vspace{1em}

Approximate the operator $q(u)$ by a linear approximation about the function $u_0$.
\begin{align*}
q(u) \approx q(u_0) + q'(u_0)(u-u_0),
\end{align*}
where $q'(u_0)$ denotes the \emph{Fr\'{e}chet} derivative.



 
\end{frame}




%-=-=-=-=-=-=-=-=-=-=-=-=-=-=-=-=-=-=-=-=-=-=-=-=
%	FRAME:
%-=-=-=-=-=-=-=-=-=-=-=-=-=-=-=-=-=-=-=-=-=-=-=-=
\begin{frame}{Iterative procedure}
First guess a solution for $u_0$, then solve
\begin{align*}
&u''_1+q'(u_0)(u_1-u_0)+q(u_0)=f(x),\\
\implies &u''_1+q'(u_0)u_1=f(x)-q(u_0)+u_0q'(u_0),
\end{align*}
subject to $u_1(a)=\alpha$ and $u_1(b)=\beta$.




\vspace{1em}
Repeating this process we get
\begin{equation*}
\begin{cases}
u''_{n+1}+q'(u_n)u_{n+1}=f(x)-q(u_n)+u_n q'(u_n),\\
u_{n+1}(a)=\alpha, u_{n+1}(b)=\beta,
\end{cases}
\end{equation*} 
until convergence.
	
\end{frame}




%-=-=-=-=-=-=-=-=-=-=-=-=-=-=-=-=-=-=-=-=-=-=-=-=
%	FRAME:
%-=-=-=-=-=-=-=-=-=-=-=-=-=-=-=-=-=-=-=-=-=-=-=-=
\begin{frame}{Iterative procedure}
Newton update at each iteration
\begin{equation*}
\begin{cases}
\delta ''_{n}+q'(u_n)\delta_n=f(x)-q(u_n)-u''_n,\\
u_{n+1} = u_n + \delta_n, \\
\delta_n(a)=0, \delta_n(b)=0.
\end{cases}
\end{equation*} 



For FEM, it is more convenient to first quasi-linearise and then discretise. 
	
\end{frame}



%-=-=-=-=-=-=-=-=-=-=-=-=-=-=-=-=-=-=-=-=-=-=-=-=
%	FRAME:
%-=-=-=-=-=-=-=-=-=-=-=-=-=-=-=-=-=-=-=-=-=-=-=-=
\begin{frame}[standout]
  End of week 4!
\end{frame}

\appendix



\end{document}
