\documentclass[10pt]{beamer}

\usetheme{metropolis}
\usepackage{appendixnumberbeamer}

\usepackage{booktabs}
\usepackage[scale=2]{ccicons}
\usepackage{pgfplots}
\usepgfplotslibrary{dateplot}

\usepackage{xspace}
\newcommand{\themename}{\textbf{\textsc{metropolis}}\xspace}

\usepackage{stmaryrd}
\usepackage{amsmath}
\usepackage{verbatim}
\usepackage{tikz}
\usepackage{siunitx}
\usepackage{color}
\usepackage[normalem]{ulem}
\usetikzlibrary{calc,decorations.pathmorphing,patterns}

%%%%%%%%%%%%%%%%%%%%%%%%%%%%%%%%%%%%%%%%%%%%%%%%%%%%%%%%%%%%%%%%%%%%%%%%%
%%%% The following is for fancy box %%%%%%%%%%%%%%
\usepackage[framemethod=TikZ]{mdframed}
%%%%%%%%%%%%% Theorem %%%%%%%%%%%%
\newenvironment{thm}[2][]{%
\ifstrempty{#1}%
{\mdfsetup{%
frametitle={%
\tikz[baseline=(current bounding box.east),outer sep=0pt]
\node[anchor=east,rectangle,fill=red!20]
{\strut Theorem};}}
}%
{\mdfsetup{%
frametitle={%
\tikz[baseline=(current bounding box.east),outer sep=0pt]
\node[anchor=east,rectangle,fill=red!20]
{\strut Theorem #1};}}%
}%
\mdfsetup{innertopmargin=1pt,linecolor=red!20,%
linewidth=2pt,topline=true,%
frametitleaboveskip=\dimexpr-\ht\strutbox\relax
}
\begin{mdframed}[]\relax%
\label{#2}}{\end{mdframed}}
%%%%%%%%%%%%%%Lemma%%%%%%%%%%%%%%%%%%%%%%%%%%%%%%%%%%%%%%%%%%%%%%%%%%%%%%
\newcounter{lem}[section] \setcounter{lem}{0}
\renewcommand{\thelem}{\arabic{section}.\arabic{lem}}
\newenvironment{lem}[2][]{%
\refstepcounter{lem}%
\ifstrempty{#1}%
{\mdfsetup{%
frametitle={%
\tikz[baseline=(current bounding box.east),outer sep=0pt]
\node[anchor=east,rectangle,fill=green!20]
{\strut Lemma~\thelem};}}
}%
{\mdfsetup{%
frametitle={%
\tikz[baseline=(current bounding box.east),outer sep=0pt]
\node[anchor=east,rectangle,fill=green!20]
{\strut Lemma~\thelem:~#1};}}%
}%
\mdfsetup{innertopmargin=1pt,linecolor=green!20,%
linewidth=2pt,topline=true,%
frametitleaboveskip=\dimexpr-\ht\strutbox\relax
}
\begin{mdframed}[]\relax%
\label{#2}}{\end{mdframed}}
%%%%%%%%%%%%%%%%%%%%%%%%%%%%%%%%%%%%%%%%%%%%%%%%%%%%%%%%%%%%%%%%%%%%%%%
%Corollary
\newenvironment{cor}[2][]{%
\ifstrempty{#1}%
{\mdfsetup{%
frametitle={%
\tikz[baseline=(current bounding box.east),outer sep=0pt]
\node[anchor=east,rectangle,fill=blue!30]
{\strut Corollary};}}
}%
{\mdfsetup{%
frametitle={%
\tikz[baseline=(current bounding box.east),outer sep=0pt]
\node[anchor=east,rectangle,fill=blue!30]
{\strut Corollary:~#1};}}%
}%
\mdfsetup{innertopmargin=1pt,linecolor=blue!30,%
linewidth=2pt,topline=true,%
frametitleaboveskip=\dimexpr-\ht\strutbox\relax
}
\begin{mdframed}[]\relax%
\label{#2}}{\end{mdframed}}
%%%%%%%%%%%%%%%%%%%%%%%%%%%%%%%%%%%%%%%%%%%%%%%%%%%%%%%%%%%%%%%%%%%%%%%
%Proof
\newenvironment{prove}[2][]{%
\ifstrempty{#1}%
{\mdfsetup{%
frametitle={%
\tikz[baseline=(current bounding box.east),outer sep=0pt]
\node[anchor=east,rectangle,fill=red!20]
{\strut Proof};}}
}%
{\mdfsetup{%
frametitle={%
\tikz[baseline=(current bounding box.east),outer sep=0pt]
\node[anchor=east,rectangle,fill=red!20]
{\strut Proof:~#1};}}%
}%
\mdfsetup{innertopmargin=1pt,linecolor=red!20,%
linewidth=2pt,topline=true,%
frametitleaboveskip=\dimexpr-\ht\strutbox\relax
}
\begin{mdframed}[]\relax%
\label{#2}}{\end{mdframed}}
%%%%%%%%%%%%%%%%%%%%%%%%%%%%%%%%%%%%%%%%%%%%%%%%%%%%%%%%%%%%%%%%%%%%%%%
%Defintion
\newcounter{drf}[section]\setcounter{drf}{0}
\renewcommand{\thedrf}{}
\newenvironment{drf}[2][]{%
\refstepcounter{drf}%
\ifstrempty{#1}%
{\mdfsetup{%
frametitle={%
\tikz[baseline=(current bounding box.east),outer sep=0pt]
\node[anchor=east,rectangle,fill=orange!20]
{\strut Definition~\thedrf};}}
}%
{\mdfsetup{%
frametitle={%
\tikz[baseline=(current bounding box.east),outer sep=0pt]
\node[anchor=east,rectangle,fill=orange!20]
{\strut Definition~\thedrf:~#1};}}%
}%
\mdfsetup{innertopmargin=1pt,linecolor=orange!20,%
linewidth=2pt,topline=true,%
frametitleaboveskip=\dimexpr-\ht\strutbox\relax
}
\begin{mdframed}[]\relax%
\label{#2}}{\end{mdframed}}
%%%%%%%%%%%%%%%%%%%%%%%%%%%%%%%%%%%%%%%%%%%%%%%%%%%%%%%%%%%%%%%%%%%%%%%%%%%
% end of self defined label
%%%%%%%%%%%%%%%%%%%%%%%%%%%%%%%%%%%%%%%%%%%%%%%%%%%%%%%%%%%%%%%%%%%%%%%%%%%


%%%%%%%%%%%%%% For double curly bracket%%%%%%%%%%%%%%%%
\usepackage{xparse}

\NewDocumentCommand{\dgal}{sO{}m}{%
  \IfBooleanTF{#1}
    {\dgalext{#3}}
    {\dgalx[#2]{#3}}%
}

\NewDocumentCommand{\dgalext}{m}{%
  \sbox0{%
    \mathsurround=0pt % just for safety
    $\left\{\vphantom{#1}\right.\kern-\nulldelimiterspace$%
  }%
  \sbox2{\{}%
  \ifdim\ht0=\ht2
    \{\kern-.45\wd2 \{#1\}\kern-.45\wd2 \}%
  \else
    \left\{\kern-.5\wd0\left\{#1\right\}\kern-.5\wd0\right\}%
  \fi
}

\NewDocumentCommand{\dgalx}{om}{%
  \sbox0{\mathsurround=0pt$#1\{$}%
  \sbox2{\{}%
  \ifdim\ht0=\ht2
    \{\kern-.45\wd2 \{#2\}\kern-.45\wd2 \}%
  \else
    \mathopen{#1\{\kern-.5\wd0 #1\{}
    #2
    \mathclose{#1\}\kern-.5\wd0 #1\}}
  \fi
}
%%%%%%%%%%%%%%%%



%%%%%%%%%%%%%%%%%%%%%%%%%%%%%%%%%%%%%

\usepackage{animate}

\graphicspath{{./fig/}}


%%%%%%%%%%%%%% define color %%%%%%%%%%%%%%%%%%
\definecolor{DarkFern}{HTML}{407428}
\definecolor{DarkCharcoal}{HTML}{4D4944}
\colorlet{Fern}{DarkFern!85!white}
\colorlet{Charcoal}{DarkCharcoal!85!white}
\colorlet{LightCharcoal}{Charcoal!50!white}
\colorlet{AlertColor}{orange!80!black}
\colorlet{DarkRed}{red!70!black}
\colorlet{DarkBlue}{blue!70!black}
\colorlet{DarkGreen}{green!70!black}

\newcommand{\I}{\mathrm{i}}




%%%%%%%%%%%%%%%%%%%%%%%%%%%%%%%%%%%%%%
%\newcommand{\G}{\alert{G}}  % change color of main author



%%%%%%%%%%%%% Title Page %%%%%%%%%%%%%%%%%%%

\title{MAST90026 Computational Differential\\Equations: Week 2}
%\subtitle{A modern beamer theme}
\date{Semester 1 2024}
\author{Jesse Collis\\Modified from Hailong Guo (2022)}
\institute{The University of Melbourne}
\titlegraphic{\vspace{6cm}\flushright\includegraphics[height=2cm]{logo.png}}


\begin{document}

\maketitle


%
%
%\begin{frame}{Table of contents}
%  \setbeamertemplate{section in toc}[sections numbered]
%  \tableofcontents[hideallsubsections]
%\end{frame}


%-=-=-=-=-=-=-=-=-=-=-=-=-=-=-=-=-=-=-=-=-=-=-=-=
%
%	SECTION: 
%
%-=-=-=-=-=-=-=-=-=-=-=-=-=-=-=-=-=-=-=-=-=-=-=-=
%\section{Method of Weighted Residuals}





%-=-=-=-=-=-=-=-=-=-=-=-=-=-=-=-=-=-=-=-=-=-=-=-=
%	FRAME:
%-=-=-=-=-=-=-=-=-=-=-=-=-=-=-=-=-=-=-=-=-=-=-=-=
\begin{frame}{Method of weighted residual }

    
    \begin{center}
  \includegraphics[width=1.03\textwidth]{wr.pdf}
     \end{center}
    

 
	
\end{frame}



%-=-=-=-=-=-=-=-=-=-=-=-=-=-=-=-=-=-=-=-=-=-=-=-=
%	FRAME:
%-=-=-=-=-=-=-=-=-=-=-=-=-=-=-=-=-=-=-=-=-=-=-=-=
\begin{frame}{Classification of MWR }

    
    \begin{center}
  \includegraphics[width=1.03\textwidth]{wrm.pdf}
     \end{center}
    

 
	
\end{frame}




%-=-=-=-=-=-=-=-=-=-=-=-=-=-=-=-=-=-=-=-=-=-=-=-=
%	FRAME:
%-=-=-=-=-=-=-=-=-=-=-=-=-=-=-=-=-=-=-=-=-=-=-=-=
\begin{frame}{Idea of spectral collocation}

   
  Three step idea:
  
  
  \begin{enumerate}
  	\item [Step 1:] Choose a trial space $V_N$ ($N+1$ dimensional). Write $u_N\in V_N$ as
     \begin{equation*}
     	u_N = \sum_{j=0}^N c_j b_j(x).
     \end{equation*} 
  	\item [Step 2:] Choose collocation points $\{x_0, x_1, \ldots, x_N\}$.
  	\item [Step 3:] Determine $c_j$ using the collocation condition:
  	\begin{equation*}
  		Lu_N(x_j) = f(x_j), \quad j =0, 1, \dots, N.
  	\end{equation*}
  \end{enumerate}



  
\end{frame}



%-=-=-=-=-=-=-=-=-=-=-=-=-=-=-=-=-=-=-=-=-=-=-=-=
%	FRAME:
%-=-=-=-=-=-=-=-=-=-=-=-=-=-=-=-=-=-=-=-=-=-=-=-=
\begin{frame}{Choose of trial space and collocation points}

Periodic boundary conditions:
\begin{itemize}
	\item Space of trigonometric functions
	\begin{equation*}
		u_N(x) = \sum_{j=0}^N a_j \cos(jx) + \sum_{j=1}^N b_j\sin(jx) = \sum_{j=-N}^{N} c_j e^{i jx}.
	\end{equation*}
	\item Collocation points: $x_j = \frac{j\pi}{N}, \quad j = 0, 1, \ldots, 2N.$
\end{itemize}


Nonperiodic boundary conditions:
  \begin{itemize}
  	\item Space of polynomials of degree $N$ (order $N+1$).
\begin{equation*}
u_N(x) =p_{N}(x) \in \mathbb{P}_{N+1}.
\end{equation*}
   \item Collocation points are roots of orthogonal polynomial of degree $N+1$.
  \end{itemize}
  
  
  
  
  
\end{frame}




%-=-=-=-=-=-=-=-=-=-=-=-=-=-=-=-=-=-=-=-=-=-=-=-=
%	FRAME:
%-=-=-=-=-=-=-=-=-=-=-=-=-=-=-=-=-=-=-=-=-=-=-=-=
\begin{frame}{Definition of orthogonal }


\begin{drf}[(Orthogonal)]
	s Given an open interval $I:=(a, b)(-\infty \leq a<b \leq+\infty)$, and a generic weight function $\omega$ such that
$$
\omega(x)>0, \forall x \in I \text { and } \omega \in L^{1}(I)
$$
two different functions $f$ and $g$ are said to be orthogonal to each other in $L_{\omega}^{2}(a, b)$ or orthogonal with respect to $\omega$ if
$$
(f, g)_{\omega}:=\int_{a}^{b} f(x) g(x) \omega(x) d x=0
$$
\end{drf}

\end{frame}


%-=-=-=-=-=-=-=-=-=-=-=-=-=-=-=-=-=-=-=-=-=-=-=-=
%	FRAME:
%-=-=-=-=-=-=-=-=-=-=-=-=-=-=-=-=-=-=-=-=-=-=-=-=
\begin{frame}{Orthogonal polynomial}


\begin{drf}
	a A sequence of polynomials $\left\{p_{n}\right\}_{n=0}^{\infty}$ with $\operatorname{deg}\left(p_{n}\right)=n$ is said to be orthogonal in $L_{\omega}^{2}(a, b)$ if
$$
\left(p_{n}, p_{m}\right)_{\omega}=\int_{a}^{b} p_{n}(x) p_{m}(x) \omega(x) d x=\gamma_{n} \delta_{m n}
$$
where the constant $\gamma_{n}=\left\|p_{n}\right\|_{\infty}^{2}$ is nonzero, and $\delta_{m n}$ is the Kronecker delta.
\end{drf}


  
\end{frame}








%-=-=-=-=-=-=-=-=-=-=-=-=-=-=-=-=-=-=-=-=-=-=-=-=
%	FRAME:
%-=-=-=-=-=-=-=-=-=-=-=-=-=-=-=-=-=-=-=-=-=-=-=-=
\begin{frame}{Existence and uniqueness of orthogonal polynomial}


\begin{thm}
s For any given positive weight function $\omega \in L^{1}(a, b)$, there exists a unique sequence of monic orthogonal polynomials $\left\{\bar{p}_{n}\right\}$ with $\operatorname{deg}\left(\bar{p}_{n}\right)=n$, which can be constructed as follows
$$
\begin{aligned}
&\bar{p}_{0}=1, \bar{p}_{1}=x-\alpha_{0} \\
&\bar{p}_{n+1}=\left(x-\alpha_{n}\right) \bar{p}_{n}-\beta_{n} \bar{p}_{n-1}, \quad n \geq 1
\end{aligned}
$$
where
$$
\begin{aligned}
\alpha_{n} &=\frac{\left(x \bar{p}_{n}, \bar{p}_{n}\right)_{\omega}}{\left\|\bar{p}_{n}\right\|_{\omega}^{2}}, \quad n \geq 0, \\
\beta_{n} &=\frac{\left\|\bar{p}_{n}\right\|_{\omega}^{2}}{\left\|\bar{p}_{n-1}\right\|_{\omega}^{2}}, \quad n \geq 1 .
\end{aligned}
$$
\end{thm}
 

  
\end{frame}


%-=-=-=-=-=-=-=-=-=-=-=-=-=-=-=-=-=-=-=-=-=-=-=-=
%	FRAME:
%-=-=-=-=-=-=-=-=-=-=-=-=-=-=-=-=-=-=-=-=-=-=-=-=
\begin{frame}{Example of orthogonal polynomials}

Consider $(a,b)=(-1,1)$ and $\omega = 1$. \alert{Legendre polynomials}:
\begin{equation*}
	P_0(x) = 1, \quad P_1(x) = x, \quad P_{n+1}(x) = \frac{2n+1}{n+1}xP_n(x) - \frac{n}{n+1}P_{n-1}(x), 
\end{equation*}

\vspace{0.2em}

Consider $(a,b)=(-1,1)$ and $\omega = (1-x^2)^{-\frac{1}{2}}$. \alert{Chebyshev polynomials of the first kind}:
\begin{equation*}
	P_0(x) = 1, \quad P_1(x) = x, \quad P_{n+1}(x) = 2xP_n(x) - P_{n-1}(x), 
\end{equation*}

\vspace{0.2em}
More general. Consider $(a,b)=(-1,1)$ and $\omega = (1-x)^{\alpha}(1+x)^{\beta}$ with $\alpha,\beta>-1$. The obtained orthogonal polynomials are \alert{Jacobi polynomials}
  
\end{frame}



%-=-=-=-=-=-=-=-=-=-=-=-=-=-=-=-=-=-=-=-=-=-=-=-=
%	FRAME:
%-=-=-=-=-=-=-=-=-=-=-=-=-=-=-=-=-=-=-=-=-=-=-=-=
\begin{frame}{Other definition of Legendre and Chebyshev polynomials}
Legendre polynomials:
\begin{itemize}
	\item Eigenfunciton of 
	$$
	-\frac{d}{dx}\left((1-x^2)\frac{d\phi(x)}{dx}\right)=k(k+1)\phi(x)
	$$
	\item Rodrigue's formula
	\begin{equation*}
		P_n(x) = \frac{1}{2^nn!}\frac{d^n}{dx^n}(x^2-1)^n.
	\end{equation*}
\end{itemize}

Chebyshev polynomial of the first kind:
\begin{itemize}
	\item Eigenfunciton of 
	$$
	-\frac{d}{dx}\left(\sqrt{(1-x^2)}\frac{d\phi(x)}{dx}\right)= \frac{k^2}{\sqrt{(1-x^2)}}\phi(x)
	$$
	\item Explicit formula
	\begin{equation*}
		P_n(x) = \cos(n\theta), \text{ where }  \theta = \arccos(x)
	\end{equation*}
\end{itemize}
  
\end{frame}





%-=-=-=-=-=-=-=-=-=-=-=-=-=-=-=-=-=-=-=-=-=-=-=-=
%	FRAME:
%-=-=-=-=-=-=-=-=-=-=-=-=-=-=-=-=-=-=-=-=-=-=-=-=
\begin{frame}{Zeros of orthogonal polynomials}

\begin{thm}
	1 The zeros of $p_{n}$ are all real, simple, and lie in the interval $(a, b)$.
\end{thm}
  
\end{frame}





%-=-=-=-=-=-=-=-=-=-=-=-=-=-=-=-=-=-=-=-=-=-=-=-=
%	FRAME:
%-=-=-=-=-=-=-=-=-=-=-=-=-=-=-=-=-=-=-=-=-=-=-=-=
\begin{frame}{Numerical quadrature}

The basis problem is to find quadrature nodes $\{x_j\}$ and weights $\{\omega_j\}$ such that 
\begin{equation*}
	\int_{a}^b f(x)\omega(x)dx \approx \sum_{j=0}^{N} f(x_j)\omega_j.
\end{equation*}

  \begin{drf}
  	f The quadrature have algebraic accuracy of degree $p$ if
  	\begin{equation*}
	\int_{a}^b p(x)\omega(x)dx = \sum_{j=0}^{N} p(x_j)\omega_j,\quad p(x) \in \mathbb{P}_p.
\end{equation*}

  \end{drf}
  
  
 In general, we want choose  $\{x_j\}$ and $\{\omega_j\}$ such that we have highest degree of accuracy.
\end{frame}




%-=-=-=-=-=-=-=-=-=-=-=-=-=-=-=-=-=-=-=-=-=-=-=-=
%	FRAME:
%-=-=-=-=-=-=-=-=-=-=-=-=-=-=-=-=-=-=-=-=-=-=-=-=
\begin{frame}{Gaussian quadrature}


\begin{thm}[(Gauss quadrature)]
s Let $\left\{x_{j}\right\}_{j=0}^{N}$ be the set of zeros of the orthogonal polynomial $p_{N+1}$. Then there exists a unique set of quadrature weights $\left\{\omega_{j}\right\}_{j=0}^{N}$, defined by (3.36), such that
$$
\int_{a}^{b} p(x) \omega(x) d x=\sum_{j=0}^{N} p\left(x_{j}\right) \omega_{j}, \quad \forall p \in P_{2 N+1}
$$
Orthogonal Polynomials and Related Approximation Results
where the quadrature weights are all positive and given by
$$
\omega_{j}=\frac{k_{N+1}}{k_{N}} \frac{\left\|p_{N}\right\|_{\omega}^{2}}{p_{N}\left(x_{j}\right) p_{N+1}^{\prime}\left(x_{j}\right)}, \quad 0 \leq j \leq N,
$$
where $k_{j}$ is the leading coefficient of the polynomial $p_{j}$.
\end{thm}


\end{frame}







%-=-=-=-=-=-=-=-=-=-=-=-=-=-=-=-=-=-=-=-=-=-=-=-=
%	FRAME:
%-=-=-=-=-=-=-=-=-=-=-=-=-=-=-=-=-=-=-=-=-=-=-=-=
\begin{frame}{Choices of collocation points}
We have many different options for collocation points that suit different scenarios


Chebyshev points:
\begin{itemize}
	\item Chebyshev-Gauss: $x_j = \cos \frac{(2j+1)\pi}{2N+2}, \quad 0\le j \le N.$
    \item Chebyshev-Gauss-Radau: $x_j = \cos \frac{2j\pi }{2N+1}, \quad 0\le j \le N.$
    \item Chebyshev-Gauss-Lobatto: $x_j = \cos \frac{j\pi }{N}, \quad 0\le j \le N.$
\end{itemize}

Legendre points:
\begin{itemize}
	\item Legendre-Gauss: $x_j$ are the zeros of $P_{N+1}(x)$.
    \item Legendre-Gauss-Radau: $x_j$ are the zeros of $P_N(x)+P_{N+1}(x)$.
    \item Legendre-Gauss-Lobatto: $x_0=-1$, $x_N=1$, $\{x_j\}_{j=1}^{N}$ are zeros of $P'_{N-1}(x)$.
   \end{itemize}

\alert{Question}: Why don't we just use equally distributed points?

\end{frame}


\begin{frame}{Chebyshev-Gauss quadrature}
The quadrature rules for Chebyshev points are particularly simple

\begin{enumerate}
\item Chebyshev-Gauss:\\$\displaystyle{\int_{-1}^{1} \frac{p(x)}{\sqrt{1-x^2}}dx = \frac{\pi}{N+1}\sum_{j=0}^{N}p(x_j),\qquad \forall p\in P_{2N+1}}$
\item Chebyshev-Gauss-Radau:\\$\displaystyle{\int_{-1}^{1} \frac{p(x)}{\sqrt{1-x^2}}dx = \frac{\pi}{2N+1}p(1) + \frac{\pi}{N+\frac{1}{2}}\sum_{j=1}^{N} p(x_j)},\qquad \forall p\in P_{2N}$
\item Chebyshev-Gauss-Lobatto:\\ $\displaystyle{\int_{-1}^{1}\frac{p(x)}{\sqrt{1-x^2}}dx = \frac{\pi}{2N}\Big(p(1) + p(-1)\Big) + \frac{\pi}{N}\sum_{j=1}^{N-1}p(x_j),\qquad \forall p\in P_{2N-1}}$
\end{enumerate}


\end{frame}



%-=-=-=-=-=-=-=-=-=-=-=-=-=-=-=-=-=-=-=-=-=-=-=-=
%	FRAME:
%-=-=-=-=-=-=-=-=-=-=-=-=-=-=-=-=-=-=-=-=-=-=-=-=
\begin{frame}{Illustration of spectral collocation method}

\alert{Example}: Consider the BVP $u''(x)+p(x)u'(x)+q(x)=r(x)$ over $[a,b]$.
\begin{itemize}
	\item   Map the interval $[a,b]$ to $[-1,1]$. $\to u''(t) +\bar{p}(t)u'(t)+\bar{q}u(t)=\bar{r}(t)$.
	Here we still denote the mapped functions as $u(t)$.
	\item Let \begin{equation*}
u_N(x) = \sum_{j=0}^{N}u_j\ell_j(x),  
\end{equation*} 
where $\ell_j(x)$ be the Lagrange interpolating polynomials. So,
\begin{equation*}
u_N'(x) = \sum_{j=0}^N u_j\ell_j'(x),
\end{equation*}
\begin{equation*}
u_N''(x)  = \sum_{j=0}^{N}u_j\ell_j''(x).
\end{equation*}
\end{itemize}

\end{frame}



%-=-=-=-=-=-=-=-=-=-=-=-=-=-=-=-=-=-=-=-=-=-=-=-=
%	FRAME:
%-=-=-=-=-=-=-=-=-=-=-=-=-=-=-=-=-=-=-=-=-=-=-=-=
\begin{frame}{Differentiation matrix}

Since we are collocating, we only need to find $u'$ and $u''$ at the collocation points.
\begin{align*}
u_N'(x_k) &= \sum_{j=0}^N u_j \ell_j'(x_k) = \sum_{j=0}^N D_{kj}u_j, \quad \text{and} \\
u_N''(x_k) &= \sum_{j=0}^N u_j \ell_j''(x_k) = \sum_{j=0}^N D^{(2)}_{kj}u_j,
\end{align*}
where $D_{kj} \equiv \frac{d}{dx}\ell_j(x)\bigg|_{x_k}$, $D_{kj}^{(2)}=\frac{d^2}{dx^2}\ell_j(x)\bigg|_{x_k}$, and $\ell_j$ is the $j$th Lagrange interpolation polynomial.  


Note that $D^{(2)} = D^2$ (squared using matrix multiplication).

\end{frame}



%-=-=-=-=-=-=-=-=-=-=-=-=-=-=-=-=-=-=-=-=-=-=-=-=
%	FRAME:
%-=-=-=-=-=-=-=-=-=-=-=-=-=-=-=-=-=-=-=-=-=-=-=-=
\begin{frame}{How to compute the Differentiation matrix }
\begin{thm}
	T The entries of $D$ are determined by
$$d_{k j}=\ell_{j}^{\prime}\left(x_{k}\right)= \begin{cases}\frac{Q^{\prime}\left(x_{k}\right)}{Q^{\prime}\left(x_{j}\right)} \frac{1}{x_{k}-x_{j}}, & \text { if } k \neq j, \\ \frac{Q^{\prime \prime}\left(x_{k}\right)}{2 Q^{\prime}\left(x_{k}\right)}, & \text { if } k=j,\end{cases}$$
where
$$
Q(x)=p_{N+1}(x),(x-a) q_{N}(x),(x-a)(b-x) z_{N-1}(x)
$$
are the quadrature polynomials  of the Gauss, Gauss-Radau and Gauss-Lobatto quadrature, respectively. We can write $p_{N+1}=(x-x_0)(x-x_1)\cdots (x-x_N)$, $q_N=(x-x_1) \cdots (x-x_N)$, and 
 $z_{N-1}=(x-x_1) \cdots (x-x_{N-1})$

\end{thm}

%\begin{prf}
%	Proof. The Lagrange basis polynomials can be expressed as
%$$
%h_{j}(x)=\frac{Q(x)}{Q^{\prime}\left(x_{j}\right)\left(x-x_{j}\right)}, \quad 0 \leq j \leq N
%$$
%Differentiating (3.75) and using the fact $Q\left(x_{j}\right)=0$ lead to
%$$
%d_{k j}=h_{j}^{\prime}\left(x_{k}\right)=\frac{Q^{\prime}\left(x_{k}\right)}{Q^{\prime}\left(x_{j}\right)} \frac{1}{x_{k}-x_{j}}, \quad \forall k \neq j .
%$$
%Applying the L'Hopital's rule twice yields
%$$
%d_{k k}=\lim _{x \rightarrow x_{k}} h_{k}^{\prime}(x)=\frac{1}{Q^{\prime}\left(x_{k}\right)} \lim _{x \rightarrow x_{k}} \frac{Q^{\prime}(x)\left(x-x_{k}\right)-Q(x)}{\left(x-x_{k}\right)^{2}}=\frac{Q^{\prime \prime}\left(x_{k}\right)}{2 Q^{\prime}\left(x_{k}\right)} .
%$$
%This completes the proof.
%\end{prf}

\end{frame}





%-=-=-=-=-=-=-=-=-=-=-=-=-=-=-=-=-=-=-=-=-=-=-=-=
%	FRAME:
%-=-=-=-=-=-=-=-=-=-=-=-=-=-=-=-=-=-=-=-=-=-=-=-=
\begin{frame}{How to compute the Differentiation matrix }

\begin{prove}
	1 The Lagrange basis polynomials can be expressed as
$$
\ell_{j}(x)=\frac{Q(x)}{Q^{\prime}\left(x_{j}\right)\left(x-x_{j}\right)}, \quad 0 \leq j \leq N
$$
Differentiating it and using the fact that $Q\left(x_{j}\right)=0$ leads to
$$
d_{k j}=\ell_{j}^{\prime}\left(x_{k}\right)=\frac{Q^{\prime}\left(x_{k}\right)}{Q^{\prime}\left(x_{j}\right)} \frac{1}{x_{k}-x_{j}}, \quad \forall k \neq j .
$$
Applying L'Hopital's rule twice yields
$$
d_{k k}=\lim _{x \rightarrow x_{k}} \ell_{k}^{\prime}(x)=\frac{1}{Q^{\prime}\left(x_{k}\right)} \lim _{x \rightarrow x_{k}} \frac{Q^{\prime}(x)\left(x-x_{k}\right)-Q(x)}{\left(x-x_{k}\right)^{2}}=\frac{Q^{\prime \prime}\left(x_{k}\right)}{2 Q^{\prime}\left(x_{k}\right)} .
$$
This completes the proof.
\end{prove}

\end{frame}




%-=-=-=-=-=-=-=-=-=-=-=-=-=-=-=-=-=-=-=-=-=-=-=-=
%	FRAME:
%-=-=-=-=-=-=-=-=-=-=-=-=-=-=-=-=-=-=-=-=-=-=-=-=
\begin{frame}{Chebyshev differentiation matrix}

This code gives the differentiation matrix for Chebyshev-Gauss-Lobatto points using Lagrange basis polynomials

  \begin{center}
  \includegraphics[width=1.03\textwidth]{cdm}
     \end{center}
    

\end{frame}







%-=-=-=-=-=-=-=-=-=-=-=-=-=-=-=-=-=-=-=-=-=-=-=-=
%	FRAME:
%-=-=-=-=-=-=-=-=-=-=-=-=-=-=-=-=-=-=-=-=-=-=-=-=
\begin{frame}{Go back to example  }

 Collocate at the collocation points and
\begin{align*}
u_N''(x_k)+\bar{p}(x_k)u_N'(x_k)+\bar{q}(x_k)u_N(x_k)=\bar{r}(x_k), \quad k=0, 1, \ldots, N,
\end{align*}
becomes
\begin{align*}
D^2\vec{u}+
\begin{pmatrix}
\bar{p}_0 & 0 & \cdots & 0 \\
0 & \bar{p}_1 & \cdots & 0 \\
\vdots & \vdots & \ddots & \vdots \\
0 & 0 & \cdots & \bar{p}_N
\end{pmatrix}
D \vec{u}
+
\begin{pmatrix}
\bar{q}_0 & 0 & \cdots & 0 \\
0 & \bar{q}_1 & \cdots & 0 \\
\vdots & \vdots & \ddots & \vdots \\
0 & 0 & \cdots & \bar{q}_N
\end{pmatrix}
\vec{u}
=
\begin{pmatrix}
\bar{r}_0 \\
\bar{r}_1 \\
 \vdots \\
\bar{r}_N
\end{pmatrix}
=\bar{\vec{r}}.
\end{align*}
And you can solve this for $\vec{u}$.



\end{frame}




%-=-=-=-=-=-=-=-=-=-=-=-=-=-=-=-=-=-=-=-=-=-=-=-=
%	FRAME:
%-=-=-=-=-=-=-=-=-=-=-=-=-=-=-=-=-=-=-=-=-=-=-=-=
\begin{frame}{Impose boundary condition }

Since we know the value of $u$ at $x=\pm1$, $(k=0,N)$, we don't need to collocate at these points. 
So we do not need the first and last rows of $D$ and $D^2$.


\vspace{0.2em}
So for the rest of our rows we have:
\begin{align*}
u'_k =& \sum_{j=0}^{N}D_{kj}u_j \qquad (k=1,\dots,N-1)\\
=&\sum_{j=1}^{N-1}D_{kj}u_j +u_0 D_{k0}+u_N D_{kN}.
\end{align*}



\end{frame}


%-=-=-=-=-=-=-=-=-=-=-=-=-=-=-=-=-=-=-=-=-=-=-=-=
%	FRAME:
%-=-=-=-=-=-=-=-=-=-=-=-=-=-=-=-=-=-=-=-=-=-=-=-=
\begin{frame}{Summary of spectral collocation method }

\alert{Pros and cons}
\begin{itemize}
	\item High-order method, easy to implement
	\item Superior to spectral-Galerkin method in dealing with variable coefficients and/or nonlinear problems
	\item Differential matrix is ill-conditioned and full
\end{itemize}



\end{frame}








%-=-=-=-=-=-=-=-=-=-=-=-=-=-=-=-=-=-=-=-=-=-=-=-=
%	FRAME:
%-=-=-=-=-=-=-=-=-=-=-=-=-=-=-=-=-=-=-=-=-=-=-=-=
\begin{frame}[standout]
  End of week 2!
\end{frame}

\appendix



\end{document}
