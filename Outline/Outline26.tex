\documentclass[11pt]{article}
                             % The preamble begins here.

\usepackage{tgpagella}
\usepackage{color}
\usepackage{xcolor}
\title{School of Mathematics and Statistics
\\ MAST90026
Computational Differential Equations \\ 2026
} 
\date{}      % Deleting this command produces today's date.
\setlength{\parindent}{0pt}
\textwidth=17.02cm
\textheight=25cm
\voffset=-3cm
\hoffset=-0.75in
\oddsidemargin 0.75in
\evensidemargin 0.75in

\renewcommand{\theenumi}{\alph{enumi}}

\newenvironment{pseudo}{\ttfamily \begin{tabbing} }{\end{tabbing}}

\setcounter{section}{0}

\begin{document}             % End of preamble and beginning of text.

\maketitle                   % Produces the title.

\part*{Course Outline}

In Computational Differential Equations you will learn how to write and implement numerical solutions to a variety of problems commonly encountered in science and engineering. Understanding the behaviour of the mathematical problem gives insight into the pitfalls for the unwary in using canned packages inappropriately or uncritically. 
\begin{itemize}
    \item The subject will cover different material to MAST30028. However, students with previous experience in MATLAB and numerical techniques will be better prepared than those without. 
    \item This year the course will cover numerical methods for two-point boundary value problems and for common partial differential equations in 1 and 2 spacial dimensions. 
    \item Topics will include: boundary value problems for ordinary differential equations; the solution of parabolic, hyperbolic and elliptic partial differential equations, finite difference, finite element and spectral methods.
\end{itemize}




\section*{Prerequisites}	
There are no formal prerequisites for this unit, however, \textcolor{red!90!black}{\textbf{the assessment for this class is in MATLAB}}, so some previous experience will be beneficial. Additionally, it is recommended that students have \textcolor{red!90!black}{\textbf{completed a subject in partial differential equations}}.

\section*{Classes}
Monday 5PM -- 6PM and Friday 11AM -- 1PM  Room 5013, Building 110 (The Spot). \textcolor{red!90!black}{\textbf{There are no desktop computers in this room. Students will need to bring a laptop/tablet with MATLAB installed or access to MATLAB online. Ensure you have a working MATLAB environment before the first class.}}
\section*{Public Holidays}
There will be no classes on the following dates:
\begin{itemize}
    \item 09/03/2026 (Labor Day) and
    \item 03/04/2026 (Good Friday).
\end{itemize}

\section*{Consultation Hours}
My consultation hours are Friday 1:30 -- 4:30PM in Room 201 in Old Geology South. 

\section*{Assessments}
\begin{itemize}
\item Four (4) weekly homework problems, worth 20\% in total (released on Friday 2PM in Weeks 1--4 and due on Friday 11:59PM in Weeks 2--5, respectively).
\begin{itemize}
    \item Homework 4 will be released a day earlier on Thursday the 26th of March and due on the Thursday the 2nd of April. This is due to the Good Friday public holiday.
\end{itemize}
\item Three (3) assignments,  worth 20\% each (released on Friday 2PM in Weeks 5, 7, 9) due on Friday 11:59PM in Weeks 7, 9, 11, respectively.
\item One (1) 15 minute group talk including a copy of slides provided to me, worth 20\%. Held in week 12 (possibly also week 11 depending on size of class).
  \end{itemize}
\textcolor{red!90!black}{\textbf{Note: Due to the late submission time, I will be uncontactable from close of business on Friday until the following Monday morning.}} Any late submissions will not be marked and awarded a grade of 0. If there are extenuating circumstances apply for an extension or special consideration (more information available in the assessment adjustments page in the student support module).

%Students need to submit a plagiarism form online (see LMS) before the first homework is returned.

\section*{Approximate Content Schedule}

\begin{center}
\begin{tabular}{|l|p{8cm}|c|c|}
\hline
Starting &  Content  & Software & Refs.   \\
\hline
2 Mar  &  BVPs I: Finite Differences (FD) & & L, I  \\
9 Mar  &  BVPs II: Collocation methods  &  & S \\
15 Mar  &   BVPs III:  Finite elements (FEM) &  & I   \\  
22 Mar  &   Handling nonlinearity &  &  \\  
29 Mar  &  Elliptic eq. I: FD & & I \\   
\cline{2-4} &   Mid semester break (1 week) &  &\\   \cline{2-4} 
12 Apr  &  Elliptic eq. II: FEM &  & G\\ 
19 Apr  & Iterative methods for linear systems &   & G, L  \\ 
26 Apr  &  Parabolic equations  I: Method of Lines (MOL) &  & L \\  
3 May  &  Parabolic equations II:  FD&  & L\\
10 May  &  Hyperbolic equations:  FD & & L  \\
17 May  &  Evolution equations :  operator splitting   & & L  \\
24 May  & Student talks  &  &  \\
\hline
\end{tabular}
\end{center}  

It is likely we will deviate from the schedule slightly due to public holidays however, the topics will be covered in this order. 
\newpage
\section*{References}
These books go beyond the subject in coverage, and somewhat in depth however they offer a good background.
\begin{itemize}
\item[G] Gockenbach, {\em Understanding and implementing the finite element method}, SIAM, 2006.
\item[I] Iserles, {\em A first course in the numerical analysis of differential equations}, 2nd ed.,  CUP, 2008.
\item[L] Leveque, {\em Finite Difference Methods for Ordinary and Partial Differential Equations: Steady-State and Time-dependent problems}, SIAM, 2007. {Note: Available online through University Libraries.}
\item[S] Jie Shen, Tao  Tang, and Li-lian Wang, {\em Spectral methods.
Algorithms, analysis and applications},
Springer Series in Computational Mathematics, 41. Springer, Heidelberg,  2011.  {Note: Available online through University Libraries.}
\end{itemize}






\section*{}
Patrick Grant

\noindent
February 2026.

\end{document}