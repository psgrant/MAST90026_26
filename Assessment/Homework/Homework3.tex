\documentclass[11pt]{article}
\usepackage{amsmath}
\usepackage{amssymb}
\usepackage{tgpagella}
\usepackage{physics}
\usepackage[colorlinks=true,urlcolor=blue,linkcolor=blue]{hyperref}
    % The preamble begins here.
\title{School of Mathematics and Statistics
\\ MAST90026
Computational Differential Equations \\ 2026
} 
% Declares the document's title.
%\author{Leslie Lamport}      % Declares the author's name.
\date{}      % Deleting this command produces today's date.

\textwidth=17.02cm
\textheight=25cm
\voffset=-3cm
\hoffset=-0.75in
\oddsidemargin 0.75in
\evensidemargin 0.75in
% Remove paragraph indent
\setlength{\parindent}{0pt}
\hypersetup{
  colorlinks, linkcolor=blue
}

%\renewcommand{\theenumi}{\alph{enumi}}
\renewcommand{\theenumii}{\roman{enumii}}

\newenvironment{pseudo}{\ttfamily \begin{tabbing} }{\end{tabbing}}

%\setcounter{section}{0}


\begin{document}             % End of preamble and beginning of text.

\maketitle                   % Produces the title.
\vspace{-20mm}
\part*{ Homework 3 \\ Due: 11:59PM Friday, 27th March.}   

This homework is worth 5\% of the total assessment in this subject. Submit your hand working and published MATLAB code* as a combined PDF file through Canvas.

\medskip
*Use the \texttt{publish} command (or use the GUI) to run your script and save it as a PDF.

\medskip
Late submissions will not be marked and a grade of 0 will be awarded. If there are extenuating circumstances apply for an extension or special consideration (more information available in the assessment adjustments page in the student support module).
\medskip

\hrule
\medskip


\begin{enumerate}
\item 
Consider the following BVP:
$$
-u^{\prime \prime}(x)+u(x)=f(x), \quad 0<x<1, \quad u(0)=u(1)=0 .
$$
Derive the linear system of the equations for the finite element approximation
$$
u_{h}=\sum_{j=1}^{3} \alpha_{j} \phi_{j}(x)
$$
with the following information:
\begin{itemize}
	\item 
	 $f(x)=1$;
\item  the nodal points and the elements are indexed as
$$
\begin{aligned}
&x_{0}=0, \quad x_{1}=\frac{1}{4}, \quad x_{2}=\frac{1}{2}, \quad x_{3}=\frac{3}{4}, \quad x_{4}=1 \\
&E_{1}=\left[x_{0}, x_{1}\right], \quad E_{2}=\left[x_{1}, x_{2}\right], \quad E_{3}=\left[x_{2}, x_{3}\right], \quad E_{4}=\left[x_{3}, x_{4}\right]
\end{aligned}
$$
\item  the basis functions are the hat functions
$$
\phi_{i}\left(x_{j}\right)= \begin{cases}1, & \text { if } i=j \\ 0, & \text { otherwise }\end{cases}
$$
\item  assemble the stiffness/mass matrix and the load vector element by element. 
\end{itemize}
What do you find the linear system to be? 
 
\item   Derive the Galerkin equations (for a given basis $\{\phi_j\}$) for the case of mixed BCs
\[ u(a)-u'(a) = \alpha, u(b) = \beta. \]

\item Write code to use finite elements with linear basis functions on uniform mesh to solve the constant coefficient Dirichlet BVP:
\[ -u'' + q u = r;\ \ u(a) = \alpha, u(b) = \beta,   \] 
 where $q, r,a,b, \alpha, \beta$ are constants.  
 
 Test your code on problem  from Homework 2:
\[ -u'' + u = 0;\ \ u(0) = 1, u(1) = \exp(1),   \] 
 and plot the maximum grid error $\max |e_j|$ versus $N$ as a log-log plot. What is the rate of convergence?
 
Submit the {\sc Matlab} code and the resulting error plot.

\end{enumerate}


 
\end{document}