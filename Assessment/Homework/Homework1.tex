\documentclass[11pt]{article}
\usepackage{amsmath}
\usepackage{amssymb}
\usepackage{tgpagella}
\usepackage[colorlinks=true,urlcolor=blue,linkcolor=blue]{hyperref}
    % The preamble begins here.
\title{School of Mathematics and Statistics
\\ MAST90026
Computational Differential Equations \\ 2026
} 
% Declares the document's title.
%\author{Leslie Lamport}      % Declares the author's name.
\date{}      % Deleting this command produces today's date.

\textwidth=17.02cm
\textheight=25cm
\voffset=-3cm
\hoffset=-0.75in
\oddsidemargin 0.75in
\evensidemargin 0.75in
% Remove paragraph indent
\setlength{\parindent}{0pt}
\hypersetup{
  colorlinks, linkcolor=blue
}

%\renewcommand{\theenumi}{\alph{enumi}}
\renewcommand{\theenumii}{\roman{enumii}}

\newenvironment{pseudo}{\ttfamily \begin{tabbing} }{\end{tabbing}}

%\setcounter{section}{0}


\begin{document}             % End of preamble and beginning of text.

\maketitle                   % Produces the title.
\vspace{-20mm}
\part*{ Homework 1 \\ Due: 11:59PM Friday, 13th March.}   

This homework is worth 5\% of the total assessment in this subject. Submit your hand working and published MATLAB code* as a combined PDF file through Canvas.

\medskip
*Use the \texttt{publish} command (or use the GUI) to run your script and save it as a PDF.

\medskip
Late submissions will not be marked and a grade of 0 will be awarded. If there are extenuating circumstances apply for an extension or special consideration (more information available in the assessment adjustments page in the student support module).
\medskip

\hrule

\medskip
\begin{enumerate}
\item 
Consider the matrix $A\in \mathbb{R}^{N\times N}$ which is  obtained by using central finite difference method to discretise $u''=f(x), u(0) = u(1) = 0$, i.e.
\begin{equation}
	A= \frac{1}{h^2}\begin{bmatrix}
		-2 & 1 & & & &\\
		1& -2 & 1 & & &\\
		& & \ddots & \ddots & \ddots & &\\
		& & & 1 & -2 &1\\
		& & & & 1 & -2 &\\
	\end{bmatrix}
\end{equation}
\begin{itemize}
	\item [(a)] Denote the $k$th eigenvalue of A by $\lambda^k$ and the corresponding eigenvalue vector is denoted by $\mathbf{u}^k$. Find $\lambda^k$ by assuming  $\mathbf{u}^k_j = \sin(\frac{k\pi j}{N+1})$, $j=1, \ldots, N;$  
	\item [(b)] Use the result in (a) to prove that $\|A^{-1}\|_2 = \frac{1}{\pi^2} + O(h^2)$.
\end{itemize}

\item  Consider the finite difference scheme for the 1D steady state convection-diffusion equation
$$
\begin{aligned}
&\epsilon u^{\prime \prime}-u^{\prime}=-1, \quad 0<x<1 \\
&u(0)=1, \quad u(1)=3
\end{aligned}
$$
(a) Verify the exact solution is
$$
u(x)=1+x+\left(\frac{e^{x / \epsilon}-1}{e^{1 / \epsilon}-1}\right)
$$
(b) Compare the following two finite difference methods for $\epsilon=0.3,0.1,0.05$, and $0.0005$.
\begin{itemize}
	\item [(i)] Central finite difference scheme:
$$
\epsilon \frac{U_{i-1}-2 U_{i}+U_{i+1}}{h^{2}}-\frac{U_{i+1}-U_{i-1}}{2 h}=-1
$$\newpage
\item [(ii)] Central-upwind finite difference scheme:
$$
\epsilon \frac{U_{i-1}-2 U_{i}+U_{i+1}}{h^{2}}-\frac{U_{i}-U_{i-1}}{h}=-1
$$
\end{itemize}
Do a grid refinement analysis for each case to determine the order of accuracy. Include plots of $\|\boldsymbol{E}\|_2$ vs $h$.

(c) From your observations, in your opinion which method is better? 

\end{enumerate}


 
\end{document}