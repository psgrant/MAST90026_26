\documentclass[11pt]{article}
\usepackage{amsmath}
\usepackage{amssymb}
\usepackage{tgpagella}
\usepackage{physics}
\usepackage[colorlinks=true,urlcolor=blue,linkcolor=blue]{hyperref}
    % The preamble begins here.
\title{School of Mathematics and Statistics
\\ MAST90026
Computational Differential Equations \\ 2026
} 
% Declares the document's title.
%\author{Leslie Lamport}      % Declares the author's name.
\date{}      % Deleting this command produces today's date.

\textwidth=17.02cm
\textheight=25cm
\voffset=-3cm
\hoffset=-0.75in
\oddsidemargin 0.75in
\evensidemargin 0.75in
% Remove paragraph indent
\setlength{\parindent}{0pt}
\hypersetup{
  colorlinks, linkcolor=blue
}

%\renewcommand{\theenumi}{\alph{enumi}}
\renewcommand{\theenumii}{\roman{enumii}}

\newenvironment{pseudo}{\ttfamily \begin{tabbing} }{\end{tabbing}}

%\setcounter{section}{0}


\begin{document}             % End of preamble and beginning of text.

\maketitle                   % Produces the title.
\vspace{-20mm}
\part*{Homework 4 \\Due: 11:59PM \textcolor{red}{Thursday, 2nd April.}}   

This homework is worth 5\% of the total assessment in this subject. Submit your hand working and published MATLAB code* as a combined PDF file through Canvas.

\medskip
*Use the \texttt{publish} command (or use the GUI) to run your script and save it as a PDF.

\medskip
Late submissions will not be marked and a grade of 0 will be awarded. If there are extenuating circumstances apply for an extension or special consideration (more information available in the assessment adjustments page in the student support module).
\medskip

\hrule
\medskip



\begin{enumerate}

\item  For the case $D(x)=1, q(x)=0$ and $\{\phi_j\}$ are the hat functions on a non-uniform mesh, show that the  stiffness matrix {\bf K} takes the form
\begin{eqnarray*}
K_{ii} & = & \frac{1}{h_i} + \frac{1}{h_{i-1}},   \\
K_{i,i-1} = K_{i-1,i} & = & -\frac{1}{h_{i-1}},  \\
K_{ij} &=& 0,\ \ \ \  |i-j|>1,
\end{eqnarray*}
i.e. {\bf K} is tridiagonal and symmetric.

\item  Write a MATLAB function \verb+FEj=elem_load(f,xj,xjp1)+
that computes, using Gaussian quadrature of order $3$, the element load vector for linear elements
\[ \mathbf{F}_{E_j} = \left[\begin{array}{c}\displaystyle\int_{x_j}^{x_{j+1}} f(x) \phi_j \ dx\\[5mm]
\displaystyle\int_{x_j}^{x_{j+1}} f(x) \phi_{j+1} \ dx\end{array}\right]. \]

You should submit your code. Hint: the Gaussian quadrature on $[0, 1]$ of order 3 is:
\begin{equation*}
    \qty(x_0,\omega_0) = \qty(-\frac{1}{2}\sqrt {\frac {3}{5}}+\frac{1}{2}, \frac{5}{18}),\quad \qty(x_1,\omega_1) = \qty(\frac{1}{2}, \frac{8}{18}),\quad \qty(x_2,\omega_2) = \qty(\frac{1}{2}\sqrt {\frac {3}{5}}+\frac{1}{2}, \frac{5}{18}).
\end{equation*}

\item  Solve the Problem 
\[ u''  - \sin(u) = -1;\ \ \ u(0)=0;\ u(1)=1, \]
using a finite difference method and newton method. Submit your code and a plot of the solution.

\end{enumerate}


 
\end{document}