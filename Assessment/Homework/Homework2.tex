\documentclass[11pt]{article}
\usepackage{amsmath}
\usepackage{amssymb}
\usepackage{tgpagella}
\usepackage{physics}
\usepackage[colorlinks=true,urlcolor=blue,linkcolor=blue]{hyperref}
    % The preamble begins here.
\title{School of Mathematics and Statistics
\\ MAST90026
Computational Differential Equations \\ 2026
} 
% Declares the document's title.
%\author{Leslie Lamport}      % Declares the author's name.
\date{}      % Deleting this command produces today's date.

\textwidth=17.02cm
\textheight=25cm
\voffset=-3cm
\hoffset=-0.75in
\oddsidemargin 0.75in
\evensidemargin 0.75in
% Remove paragraph indent
\setlength{\parindent}{0pt}
\hypersetup{
  colorlinks, linkcolor=blue
}

%\renewcommand{\theenumi}{\alph{enumi}}
\renewcommand{\theenumii}{\roman{enumii}}

\newenvironment{pseudo}{\ttfamily \begin{tabbing} }{\end{tabbing}}

%\setcounter{section}{0}


\begin{document}             % End of preamble and beginning of text.

\maketitle                   % Produces the title.
\vspace{-20mm}
\part*{ Homework 2 \\ Due: 11:59PM Friday, 20th March.}   

This homework is worth 5\% of the total assessment in this subject. Submit your hand working and published MATLAB code* as a combined PDF file through Canvas.

\medskip
*Use the \texttt{publish} command (or use the GUI) to run your script and save it as a PDF.

\medskip
Late submissions will not be marked and a grade of 0 will be awarded. If there are extenuating circumstances apply for an extension or special consideration (more information available in the assessment adjustments page in the student support module).
\medskip

\hrule
\medskip

\begin{enumerate}
\item 
Write a code  to solve the mixed BC problem
\[ u'' + p u' + q u = r;\ u(a) - u'(a)= \alpha,\ u(b) = \beta,    \] 
 where $p, q, r, a, b, \alpha, \beta$ are constants.
 
 Use
  \begin{itemize}
\item [(i)] a first order FD formula
\item [(ii)] a 2nd order method
\end{itemize}
to handle the Robin BC at $x=a$.

Test your code on problem 
\[ u'' - u = 0;\ \ u(0)-u'(0) = 0 , u(1)  = \exp(1),  \] 
 whose exact solution is $u(x) = \exp(x) $ and plot the maximum grid error $\max |e_j|$ versus $N$ 
as a log-log plot, for each method.

%\item  Solve problem A from Homework 1 by hand using collocation with 1 collocation point at $x=1/2$ and the 2 BCs 
%(3 conditions) to determine the polynomial of order 3 (quadratic)
% \[ u(x) \approx c_0+c_1 x+c_2 x^2. \]
% Plot your approximate answer versus the exact answer.
 
 
\item  Write code to use collocation at Legendre Gaussian Lobatto points to solve the constant coefficient Dirichlet BVP:
\[ u'' + p u' + q u = r;\ \ u(-1) = \alpha, u(1) = \beta,   \] 
 where $p, q, r, \alpha, \beta$ are constants.  

\begin{itemize}
	\item [(a)]
 Test your code on   problem A:
\[ u'' - u = 0;\ \ u(-1) = 1, u(1) = 3,   \] 
  whose exact solution is $\hat{u}(x) = 2\cosh (x) /\cosh (1)+\sinh (x) /\sinh (1)$. Inspect the solution visually by plotting both the numerical and exact solution on the same axes.
  
  \item[(b)] Approximate the error using $\lVert E\rVert_2 = \sqrt{\int_{-1}^{1} (u - \hat{u})^2dx }$ and an appropriate quadrature rule. Plot the error vs $N$. 
 

 \item [(b)]  What change would you have to make to handle the problem with  
 $u(a)=\alpha, u(b)=\beta$.
 \end{itemize} 
\end{enumerate}




 
\end{document}